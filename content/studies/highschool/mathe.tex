\documentclass[a4paper]{article}
\usepackage{german, graphicx, epsfig, subfigure}
\setlength{\parindent}{0em}
\setlength{\topmargin}{0in}
\setlength{\headheight}{0in}
\setlength{\headsep}{0in}
%\setlength{\textwidth}{6.0in}
\setlength{\textheight}{9.5in}
\setlength{\unitlength}{2.3cm}
\usepackage[german]{mymacros_goerz}
\usepackage[bookmarks=true,colorlinks=true]{hyperref}

\title{Zusammenfassung Vektorrechnung und Komplexe Zahlen}
\author{Michael Goerz}

\begin{document}

\maketitle
\tableofcontents

\section{Vektoren, Geraden und Ebenen}

\subsection{L"ange eines Vektors}
Die L"ange eines Vektors ist die Wurzel aus der Summe der Quadrate seiner Koordinaten.
$$\vec{a} = \vecttt{a_1}{a_2}{a_3} \;\Leftrightarrow\; \vert \vec{a} \vert = \sqrt{a_1^2 + a_2^2 + a_3^2}$$
Bsp.:
$$\vec{a} = \vecttt{3}{-4}{7} \;\Leftrightarrow\; \vert \vec{a} \vert = \sqrt{(3)^2 + (-4)^2 + (7)^2} = \sqrt{74} = 8.607$$
\paragraph{Einheitsvektor:}
Ein Einheitsvektor ist ein Vektor der L"ange 1.
$$a_0 = \frac{1}{|\vec{a}|} \cdot \vec{a}$$
Bsp.:
$$a_0 = \frac{1}{\sqrt{74}} \cdot \vecttt{3}{-4}{7} = \frac{\sqrt{74}}{74} \cdot \vecttt{3}{-4}{7}$$

\subsection{Skalarprodukt}
$$\vec{a} = \vecttt{a_1}{a_2}{a_3}, \qquad \vec{b} = \vecttt{b_1}{b_2}{b_3}$$
$$\vec{a} \cdot \vec{b} = a_1b_1 + a_2b_2 + a_3b_3$$
Bsp.:
$$\vec{a} = \vecttt{-3}{7}{-5}, \qquad \vec{b} = \vecttt{-7}{2}{1}$$
$$\vec{a} \cdot \vec{b} = (-3)(-7) + 7 \cdot 2 + (-5) \cdot 1 = 30$$
\paragraph{Winkel zwischen zwei Vekoren}
\begin{eqnarray*}
\cos{\alpha} &=& \frac{\vec{a} \cdot \vec{b}}{|\vec{a}| \cdot |\vec{b}|}\\
             &=& \frac{a_1b_1 + a_2b_2 + a_3b_3}{\sqrt{a_1^2 + a_2^2 + a_3^2}\cdot\sqrt{b_1^2 + b_2^2 + b_3^2}}
\end{eqnarray*}
Bsp.:
\begin{eqnarray*}
\cos{\alpha} &=& \frac{\vec{a} \cdot \vec{b}}{|\vec{a}| \cdot |\vec{b}|} \\
             &=& \frac{(-3)(-7) + 7 \cdot 2 + (-5) \cdot 1}{\sqrt{(-3)^2 + (7)^2 + (-5)^2}\cdot\sqrt{(7)^2 + (2)^2 +(1)^2}}\\
 \;\Rightarrow\; \alpha &=& 63.38\degree
\end{eqnarray*}

\subsection{Darstellungen der Ebene}
\subsubsection{Normalenform der Ebene}
Die Normalenform einer Ebene lautet: $E: \; \left[ \vec{x} - \vec{p} \right] \cdot \vec{n} = 0$.
\paragraph{Normalenform aus Parametergleichung}
\begin{itemize}
\item Bestimmung des Normalenvektors $\vec{n}$.
\item Aufstellung der Normalenform.
\end{itemize}
\begin{eqnarray*}
E:\; \vec{x} &=& \vec{p} + r\cdot\vec{u} + s\cdot\vec{v}
\end{eqnarray*}
\begin{eqnarray*}
\text{I) }\vec{u} \cdot \vec{n} = 0\\
\text{II) }\vec{v} \cdot \vec{n} = 0\\
\end{eqnarray*}



Bsp.:
\begin{eqnarray*}
E:\; \vec{x} &=& \vecttt{2}{4}{3} + r\cdot\vecttt{1}{1}{1} + s\cdot\vecttt{2}{-1}{4}
\end{eqnarray*}
\begin{eqnarray*}
\text{I) } n_1 + n_2 +n_3 = 0\\
\text{II) } 2n_1 - n_2 + 4n_3 = 0
\end{eqnarray*}
\begin{eqnarray*}
\dots\\
n_2 &=& \frac{2}{3} \; n_3, \quad n_3 \text{ sei gleich 3 } \\
\Rightarrow n_2 &=& 2\\
\Rightarrow\; n_1 &=& -5\\
\vec{n} &=& \vecttt{-5}{2}{3}
\end{eqnarray*}

\begin{eqnarray*}
E:\; \left[ \vec{x} - \vecttt{2}{4}{5} \right]-\vecttt{-5}{2}{3} = 0
\end{eqnarray*}


\paragraph{Normalenform aus Koordinatenform}
\begin{itemize}
\item Normalenvektor kann abgelesen werden ($E:\; n_1x_1 + n_2 x_2 + n_3x_3+r=0$)
\item Punkt der Ebene bestimmen ($x_1$ und $x_2$ festlegen, $x_3$ bestimmen)
\item Aufstellung der Normalenform
\end{itemize}

\paragraph{Normalenform aus Gerade und Punkte}
\begin{itemize}
\item Differenzvektor $\vec{QP}$ bilden
\item $\vec{n}$ aus $\vec{u} \cdot \vec{n} = 0$ und $\vec{PQ} \cdot \vec{n} = 0$ bestimmen
\item Aufstellung der Normalenform
\end{itemize}


\paragraph{Normalenform aus drei Punkten}
\begin{itemize}
\item Zwei Differenzvektoren bilden
\item $\vec{n}$ bestimmen
\item Aufstellung der Normalenform
\end{itemize}

\paragraph{Normalenform aus zwei parallelen Geraden}
\begin{itemize}
\item Differenzvektor $\vec{PQ}$ der St"utzvektoren bestimmen
\item $\vec{n}$ bestimmen
\item Aufstellung der Normalenform
\end{itemize}

\paragraph{Normalenform aus zwei Geraden}
\begin{itemize}
\item $\vec{n}$ aus den beiden Richtungsvektoren bestimmen
\item Aufstellung der Normalenform
\end{itemize}


\subsubsection{Parameterform aus Koordinatenform}
Die Parameterform einer Ebene lautet $E:\; \vec{x} = \vec{p} + r \cdot \vec{u} + s \cdot \vec{v}$.
\begin{itemize}
\item Drei Punkte suchen
\item $\vec{u}$ und $\vec{v}$ sind Differenzvektoren verschiedener Punkte
\item Aufstellung der Parameterform
\end{itemize}
Bsp.:
$$ x_1 - x_2 + 2x_3 = 1 \;\Leftrightarrow \; x_3 = \frac{1}{2}(1-x_1+x_2) $$
$$ A(2|1|0), \; B(4|1|1), \; C(1|0|0) $$
$$ \vec{u} = \vec{AB} = \vecttt{2}{0}{1}, \; \vec{v} =\vec{CB}=\vecttt{3}{1}{1} $$
$$ E:\; \vec{x}=\vecttt{1}{0}{0} + r \cdot \vecttt{2}{0}{1} + s \cdot \vecttt{3}{1}{1} $$

\subsubsection{Koordinatenform aus Normalenform}
Man erh"alt die Koordinatenform durch ausmultiplizieren der Normalenform.
$$E:\; \left[ \vecttt{x_1}{x_2}{x_3} - \vecttt{p_1}{p_2}{p_3} \right] \cdot \vecttt{n_1}{n_2}{n_3} = 0$$
$$E:\; n_1x_1 + n_2x_2 + n_3x_3 - (p_1n_1+p_2n_2+p_3n_3)$$
Bsp.:
$$E:\; \left[ \vec{x} - \vecttt{1}{2}{1} \right] \cdot \vecttt{2}{3}{-1} = 0$$
$$E:\; 2 x_1 + 3 x_2 -x_3 - (2+6-1) = 2 x_1 + 3 x_2 - x_3 - 7 = 0$$

\subsection{Orthogonalit"at von Geraden und Ebenen}
\paragraph{zwei Geraden}
Zwei Geraden sind orthogonal wenn ihre Richtungsvektoren orthogonal sind. Dies gilt auch f"ur windschiefe Geraden.\\
Bsp:
$$g_1: \; \vecttt{1}{2}{5} + r \cdot \vecttt{3}{0}{-2}; \quad g_2:\; \vecttt{5}{6}{1} + r \cdot \vecttt{4}{-1}{6}$$
$$ \vec{u_1} \cdot \vec{u_2} = 12 + 0 - 12 = 0 \quad \Rightarrow g_1 \text{ und } g_2 \text{ sind orthogonal}$$

\paragraph{zwei Ebenen}
Zwei Ebenen sind orthogonal wenn ihre Normalenvektoren orthogonal sind.

\paragraph{eine Gerade und eine Ebene}
Eine Gerade und eine Ebene sind orthogonal, wenn der Normalenvektor und der Richtungsvektor der Geraden linear abh"angig sind.

\subsection{Achsenschnittpunkte}
Man berechnet den Schnittpunkt mit der
\begin{itemize}
\item $x_1$-Achse, indem man $x_2 = x_3 = 0$ setzt.
\item $x_2$-Achse, indem man $x_1 = x_3 = 0$ setzt.
\item $x_3$-Achse, indem man $x_1 = x_2 = 0$ setzt.
\end{itemize}

Bsp.:
$$E:\; 5x_1 - 4x_2 +x_3 =9$$
$$S_1 (9/5|0|0)$$
$$S_2 (0|-9/4|0)$$
$$S_3 (0|0|9)$$

\subsection{Lagebestimmung von Gerade und Ebene}
\paragraph{Zwei Geraden: }
Man ermittelt die Lagebeziehung zweier Geraden, indem man ihre Gleichungen gleich setzt. Zwei Geraden
\begin{itemize}
\item schneiden sich bei einer L"osung des LGS.
\item sind gleich bei unendlich vielen L"osungen des LGS
\item sind parallel bei keiner L"osung des LGS und wenn die Richtungsvektoren linear abh. sind.
\item sind windschief bei keiner L"osung des LGS und wenn die Richtungsvektoren linear unabh. sind.
\end{itemize}

\paragraph{Zwei Ebenen}
\begin{itemize}
\item sind gleich, wenn ihre Koordinatengleichungen gleich sind.
\item sind parallel, wenn ihre Normalenvektoren linear abh"angig sind.\
\item schneiden sich, wenn ihre Normalenvektoren linear unabh"angig sind. F"ur den Schnittwinkel gilt $$\cos(\alpha) = \frac{\vec{n_1} \cdot \vec{n_2}}{|\vec{n_1}| \cdot |\vec{n_2}}$$
\end{itemize}


\paragraph{Eine Gerade und eine Ebene:}
Man ermittelt die Lagebeziehung einer Gerade und einer Ebene, indem die Geradengleichung koordinatenweise in die Koordinatengleichung der Ebene einsetzt. Die Gerade
\begin{itemize}
\item liegt auf der Ebene, wenn sich eine allgemeing"ultige Aussage ergibt.
\item ist parallel zur Ebene, wenn sich eine falsche Aussage ergibt.
\item schneidet die Ebene in dem Punkt mit dem Parameter $r$, wenn sich f"ur $r$ ein konkreter Wert ergibt. F"ur den Winkel zwischen Gerade und Ebene gilt $$\sin(\alpha)= \frac{\vec{n} \cdot \vec{u}}{|\vec{n}| \cdot |\vec{u}}$$
\end{itemize}
Bsp.:
$$g:\; \vec{x} = \vecttt{1}{2}{-1} + r \cdot \vecttt{-1}{6}{2}$$
$$E:\; 2x_1+x_2 -3x_3 = 13$$
$$2(1-r) (2+6r) - 3(-1+2r) = 13 \; \Rightarrow\; -2r + 7 = 13 \;\Rightarrow\; r=-3$$
$$S(4|-16|-7)$$
$$\alpha = 4,79\degree$$





\section{Abst"ande}
\subsection{Abstand eines Punktes von einer Ebenen}
Den Abstand eines Punktes von einer Geraden ermittelt man mit Hilfe der Hesseschen Normalenform
$$d = (\vec{x} - \vec{p}) \quad \Leftrightarrow \quad d = \frac{ax_1 + bx_2 + cx_3 -r}{\sqrt{a^2+b^2+c^2}}$$
F"ur $x$ werden die Koordinaten des Punktes eingesetzt, z.Bsp.:
$$E: \; 3x_1 -x_2+5x_3 = 9; \qquad R: (8 | 4| 1)$$
$$d = \frac{3 \cdot 8 - 1 \cdot 4 + 5\cdot 1 -9}{\sqrt{3^2+(-1)^2}+5^2}=2.7$$

\subsection{Abstand eines Punktes $P$ von einer Geraden $g$ im $\sets{R}^2$}
\begin{itemize}
\item Bestimme eine Gerade $h$, die senkrecht auf $g$ steht und durch den Punkt $P$ geht.
\item Bestimme den Schnittpunkt $S$ von $g$ und $h$.
\item Bestimme die L"ange des Vektors $\vec{SP}$.
\end{itemize}
Bsp.:
$$g: \; \vec{x} = \vectt{1}{2} + r \cdot \vectt{3}{-1}; \qquad P: \; (4|9)$$
$$H: \; \vec{x} = \vectt{4}{9} + s \cdot \vectt{1}{3}$$
$$g=h \;\Rightarrow\; s = -2.4\;\Rightarrow\;r=0.2\;\Rightarrow\;S(1.6|1.8|)$$
$$d=\sqrt{(4-1.6)^2+(9-1.8)^2} = 7.59$$

\subsection{Abstand eines Punktes $P$ von einer Geraden $g$ im $\sets{R}^3$}
\begin{itemize}
\item Bestimme eine Ebene $E$, die senkrecht zu $g$ ist und durch den Punkt $P$ geht.
\item Bestimme den Schnittpunkt $S$ von $g$ und $E$ durch koordinatenweises Einsetzen von $g$ in die Koordinatengleichung von $E$.
\item Bestimme den Betrag des Vektors $\vec{SP}$.
\end{itemize}
Bsp.:
$$g: \; \vec{x} = \vecttt{1}{3}{0} + r \cdot \vecttt{2}{-1}{1}; \qquad P: \; (5|-1|2)$$
$$E: \; \left[ \vec{x} - \vecttt{5}{-1}{2} \right] \cdot \vecttt{2}{-1}{1} = 0 \;\Leftrightarrow\; 2x_1 - x_2 + x_3 -13 = 0$$
$$2(1+2r) - (3-r) + r-13 = 0  \;\Leftrightarrow\; r = 2.33 \; \Rightarrow \; S = \left( 5.67 | 0.67| 2.33 \right)$$
$$d = \sqrt{\left(5-5.67\right)^2 + \left( (-1) - 0.67 \right)^2 + \left(2-2.33 \right)^2} = 1.83$$

\section{Komplexe Zahlen}
\subsection{Darstellung}
\begin{center}
% use packages: array
\begin{tabular}{|p{.5\textwidth}  | p{.5\textwidth}|}
\hline
Summenform & Polarform \\
\hline
\begin{eqnarray*}
z &=& a+i \cdot b
\end{eqnarray*} &
\begin{eqnarray*}
z &=& r \cdot ( \cos \alpha + i \cdot \sin \alpha) \\
  &=& r \cdot \text{cis } \alpha
\end{eqnarray*}\\
\hline
\end{tabular}
\end{center}

\renewcommand{\cis}{\text{cis }}
\subsection{Rechenarten}
\begin{tabular}{| p{0.5\textwidth} | p{0.5\textwidth} |}
\hline
\begin{eqnarray*}
z_1 &=& a + b \cdot i\\
z_2 &=& c + d \cdot i
\end{eqnarray*} &
\begin{eqnarray*}
z_1 = r \cdot \cis \alpha\\
z_2 = s \cdot \cis \beta
\end{eqnarray*}\\
\hline
\end{tabular}

\subsubsection{Addition}
\begin{tabular}{| p{0.5\textwidth} | p{0.5\textwidth} |}
\hline
\begin{eqnarray*}
z_1 + z_2 = (a + c) + (b+d) \cdot i
\end{eqnarray*} & \\
\hline
\end{tabular}

\subsubsection{Subtraktion}
\begin{tabular}{| p{0.5\textwidth} | p{0.5\textwidth} |}
\hline
\begin{eqnarray*}
z_1 - z_2 = (a - c) + (b-d) \cdot i
\end{eqnarray*} & \\
\hline
\end{tabular}


\subsubsection{Multiplikation}
\begin{tabular}{| p{0.5\textwidth} |  p{0.5\textwidth} |}
\hline
\begin{minipage}[t]{0.5\textwidth}
    \begin{eqnarray*}
    z_1 \cdot z_2 &=& (a + b \cdot i) \cdot (c + c \cdot i)\\
     &=& ( a \cdot c - b \cdot d) \\
     && + (a \cdot d + b \cdot c) \cdot i\\
    \end{eqnarray*}
\end{minipage}
&
\begin{minipage}[t]{0.5\textwidth}
    \begin{eqnarray*}
    z_1 \cdot z_2 = r \cdot s \cdot \cis(\alpha + \beta)\\
    \end{eqnarray*}
\end{minipage}\\
\hline
\end{tabular}

\subsubsection{Division}
\begin{tabular}{| p{0.5\textwidth} |  p{0.5\textwidth} |}
\hline
\begin{minipage}[t]{0.5\textwidth}
    \begin{eqnarray*}
    \frac{z_1}{z_2} &=& \frac{a+b \cdot i}{c + d \cdot i} = \frac{a+b\cdot i}{c+d\cdot i} \cdot \frac{c-d\cdot i}{c-d\cdot i}\\
     &=& \frac{(a \cdot c + b \cdot d) + (b \cdot c - d \cdot a) \cdot i}{c^2 + d^2}\\
    \end{eqnarray*}
\end{minipage}
&
\begin{minipage}[t]{0.5\textwidth}
    \begin{eqnarray*}
    \frac{z_1}{z_2} &=& \left( \frac{r}{s} \right) \cdot \cis (\alpha - \beta)\\
    \end{eqnarray*}
\end{minipage}\\
\hline
\end{tabular}

\subsection{Berechnung von Quadratwurzeln}
\begin{eqnarray*}
z &=& \sqrt{r \cdot \cis(\alpha)}
\end{eqnarray*}
hat die beiden L"osungen
\begin{eqnarray*}
z_1 &=& \sqrt{r} \cdot \cis\left( \frac{\alpha}{2} \right)\\
z_2 &=& \sqrt{r} \cdot \cis\left( \frac{\alpha}{2} + 180\degree \right)
\end{eqnarray*}


\end{document}
