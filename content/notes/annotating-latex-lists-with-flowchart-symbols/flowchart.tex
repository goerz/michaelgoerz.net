\documentclass[a4paper,10pt]{article}
\usepackage{german}
\usepackage{tikz}
\usetikzlibrary{calc}
\usepackage[qm]{mymacros_goerz}
\usepackage[utf8]{inputenc}
\usepackage[
      pdftitle={Flowchart List},
      pdfauthor={Michael Goerz},
      colorlinks=true,
      linkcolor=black,
      urlcolor=black,
      citecolor=black,
      bookmarksopen=false,
      breaklinks=true,
      plainpages=false,
      pdfpagelabels
]{hyperref}



\title{Flowchart List Demo}
\author{Michael Goerz}
\date{\today}

\newcommand\fcInstr[1]{%
  \begin{tikzpicture}[x=10pt, y=10pt, remember picture]%
    \node[coordinate] (#1) at (1,0.5) {};
    \draw (0,0) rectangle (2,1);%
  \end{tikzpicture}%
}

\newcommand\fcLoopStart[1]{%
  \begin{tikzpicture}[x=10pt, y=10pt, baseline, remember picture]%
    \node[coordinate] (#1) at (1,0.5) {};%
    \draw (0,0) -- (0,0.5) -- (0.5,1) -- (1.5,1) -- 
          (2,0.5) -- (2,0) -- (0,0) -- cycle;
  \end{tikzpicture}%
}

\newcommand\fcLoopEnd[1]{%
  \begin{tikzpicture}[x=10pt, y=10pt, baseline, remember picture]%
    \node[coordinate] (#1) at (1,0.5) {};
    \draw (0.5,0) -- (0,0.5) -- (0.0,1) -- (2,1) -- 
          (2,0.5) -- (1.5,0) -- (0.5,0) -- cycle;
  \end{tikzpicture}%
}

\begin{document}

\maketitle

\section{Implementation of the New Algorithm}

\begin{itemize}
%\setlength{\itemsep}{-2pt}
  \item[\fcInstr{a}] Set $t_n$: time grid for global $[0,T]$
  \item[\fcInstr{b}] Set $\tau_j$: time grid for short-time $[t_n,t_{n+1}]$
  \item[\fcInstr{c}] 
     $i\frac{\partial}{\partial\tau}\ket{\Psi^0(\tau)} 
      = \op{H}_n\ket{\Psi^0(\tau)}$ with
     $\ket{\Psi^0(\tau=t_n)} = \ket{\Psi(t_n)}$;\\
     $\tau \in [t_n,t_{n+1}]$
  \item[\fcLoopStart{d}] Solve inhomog. TDSE iteratively:
  \item[\fcInstr{e}] Evaluate $\ket{\Phi(\tau)}$ at sampling points and
     calculate expansion coefficients of inhomog. term by a cosine 
     transformation
  \item[\fcInstr{f}] Determine $\ket{\lambda_{(j)}}$ from prev. iter. and
     $\op{F}_m \ket{\lambda^{(m)}}$ from Cheby recursion
  \item[\fcLoopEnd{g}] Vanishing inhomog. term: 
     $\left\|\ket{\Psi^{k-1}(t_{n-1})}-\ket{\Psi^k(t_{n+1})}\right\| < \epsilon$
  \item[\fcInstr{h}] Evaluate $\Psi(t_{n+1}) = \ket{\Psi^k(t_{n+1})}$
\end{itemize}

% Draw lines between boxes
\begin{tikzpicture}[remember picture, overlay]
 \draw[->] ($(a)+(0,-5pt)$) -- ($(b)+(0,5pt)$);
 \draw[->] ($(b)+(0,-5pt)$) -- ($(c)+(0,5pt)$);
 \draw[->] ($(c)+(0,-5pt)$) -- ($(d)+(0,5pt)$);
 \draw[->] ($(d)+(0,-5pt)$) -- ($(e)+(0,5pt)$);
 \draw[->] ($(e)+(0,-5pt)$) -- ($(f)+(0,5pt)$);
 \draw[->] ($(f)+(0,-5pt)$) -- ($(g)+(0,5pt)$);
 \draw[->] ($(g)+(0,-5pt)$) -- ($(h)+(0,5pt)$);
 \draw[->] ($(g)+(-10pt,0)$)-- +(-6pt, 0) |- ($(d)+(-10pt,0)$);
\end{tikzpicture}

\end{document}
